\documentclass[journal]{IEEEtran}
\usepackage{cite}
\usepackage[pdftex]{graphicx}
\usepackage[cmex10]{amsmath}
\usepackage{amssymb}
\interdisplaylinepenalty=2500
\usepackage{array}
\usepackage{dblfloatfix}

%%%Undefined packages
% \usepackage{todonotes}
% \usepackage{changes}
% \definechangesauthor[name={Min Tan} color=orange]{MT}

\begin{document}
\title{A Single-Control-Circuit Multiple-Output Digital Low Dropout Regulator With Delay Switching}
\author{
        Kaixuan~Ye,~\IEEEmembership{Student Member,~IEEE,}
        Min~Tan,~\IEEEmembership{Member,~IEEE,}% <-this % stops a space
\thanks{The authors are with the School of Optical and Electronic Information,
Huazhong University of Science and Technology, Wuhan 430074, China (email:
mtan@hust.edu.cn).}
        }

\markboth{Journal of \LaTeX\ Class Files,~Vol.~14, No.~8, August~2015}%
{K.~Ye\MakeLowercase{\textit{et al.}}: DLDO Paper Draft}


\maketitle

\begin{abstract}
This paper presents a novel single-control-circuit multiple-output digital low dropout regulator (DLDO). Operating in the time division multiplexing mode, this DLDO can share a single control circuit and regulate multiple outputs at different time period. With the proposed delay switching technique, the cross regulation between different outputs can be totally eliminated. A prototype with two outputs was fabricated with UMC 130nm process and this approach can be extended to more outputs theoretically. As the DLDO design becomes more and more complicate and the control circuit takes up more and more area, this design points out a very promising way to save the total chip area.\\

\begin{IEEEkeywords}
Delay switching, time-division multiplexing, digital low dropout regulator.
\end{IEEEkeywords}
\end{abstract}

\section{Introduction}
In today's System on Chip (SoC) design, it usually requires multiple power domains to obtain the maximum energy efficiency for different application demands \cite{original,AALDO,AALDO1,coarse-fine,pipeline,asynchrounous,recursive,AP}. To get a highly-efficient, highly-accurate power domain, the most popular choice is to apply the hierarchical power managing network. A basic hierarchical power managing network is depicted in Fig.\ref{hierarchical}. Firstly, the DC-DC buck converters scale down the battery voltage to the operating voltage of the application with high efficiency. Then, a post regulator, most frequently a low dropout regulator (LDO), is connected with the DC-DC to reduce the output voltage ripple. Finally, the fine-grained voltage is supplied for different function units such as RFs, AD/DAC, I/Os in the circuit.

Traditionally, the LDOs are analog circuits which are implemented with error amplifiers and power stages. However, with the SoC supply voltage constantly scaling down, it becomes harder and harder for the amplifiers to meet the performance requirements. And LDO implemented with digital logics, or DLDO, draws great attention in academic recently. The DLDO was first proposed in \cite{original}. It consists of a clocked comparator, a bidirectional shift register and a PMOS array. The comparator compares the output with the reference. When the output is below the reference, the shift register shift right and turn on a unit of the PMOS array at every clock rising edge or vice versa. When the current provided by the PMOS array meet the load current, the output is regulated to the around of the reference.

However, the original DLDO suffer from large drop voltage and ripple. And many advanced techniques have been proposed to address these problems. In\cite{coarse-fine}, two MOS arrays with different size are employed. In steady state, the small MOS array is connected to the loop, and it switches to the large MOS array when the load changes. As a result, a relatively small drop voltage and ripple is realized simultaneously. In \cite{AALDO,NANDbasedAAloop}, an analog-assisted (AA) loop composing of coupling capacitor and resistor is added. The AA loop sense the output undershoot and feed it back to the buffer of the PMOS array. Consequently, the PMOS array can response to the output before the clock rising edge, so greatly reduce the undershoot. In \cite{pipeline}, it adopts a pipeline control structure and a 3-D power stage to realize the multi-step switching scheme. As a result it speeds up the transient response while maintaining a small output voltage ripple. And in \cite{recursive}, it proposed a recursive DLDO which improves the performances by applying a SAR-like binary search algorithm and a sub-LSB pulse width modulation duty control scheme. 
\begin{figure}[t!]
    \centering
    \includegraphics{pic/DLDOsche/SOCcircuit.pdf}
    \caption{Hierarchical power managing network in SoC}
    \label{hierarchical}
\end{figure}

Although all these techniques are effective in some way, with the control circuit becoming more and more complicated, the area took up by the control circuit become equivalent and even greater than the power stage. So it becomes a very interesting topic how we can improve the area efficiency of the control circuit. To improve chip area efficiency, the time-division-multiplexing (TDM) scheme has been applied in many other circuits and has achieved great success. As is a digital concept, this scheme should be available to the DLDO design as well.

The remainder of this paper is organized as follow. In section II, we discuss the TDM scheme in other circuit, and address a method to apply it to DLDO. In section III, we present the architecture and working principles of the proposed TDM DLDO design. In section IV, we interpret the circuit implementation and design considerations. In section V, we show the measurement results along with comparisons. In section VI, we conclude our proposed design.
%%This paragragh needs further tuning.
\section{TDM scheme}
In some circuit designs, a certain building block takes up a great percentage of the whole chip area. Instances are like the controller in the micro ring modulator, the inductor in the DC-DC converter, and the compensation capacitor in the low dropout regulator (LDO). Normally, more than one such circuits are needed in a SoC circuit, and the most straightforward way to realize it is to implement N single output such circuits. However, one the one hand, this immoderately consumes the chip area, and on the other hand, passive devices like inductor and capacitor may bring about great noise to the circuit. To solve this problem, the TDM scheme has been applied to many circuits and have achieved great success.
\subsection{TDM micro ring modulator}
In the burgeoning integrated optoelectronic design, the silicon Micro Ring Modulator (MRM) is believed to be the most important device to realize optical interconnect, which have been proposed to displace electrical interconnects as the next generation I/O links due to its ultra-wide bandwidth and low loss advantages \cite{wangzhicheng}. However, as is highly sensitive to the thermal fluctuation, the MRM usually requires a feedback control circuit to stabilize its resonant wavelength. Normally, the control circuit is much larger than the micro ring, and in order to save chip area, \cite{wangzhicheng} have proposed a TDM scheme to simultaneous lock the wavelength of multiple silicon micro rings.
\begin{figure}[t!]
    \centering
    \includegraphics[width=\linewidth]{pic/TDM/MRM.pdf}
    \caption{The TDM scheme applying in micro ring design}
    \label{fig:MRM}
\end{figure}

Fig.\ref{fig:MRM} shows the working principle of the TDM scheme. The MUX and DEMUX connect the photo detector(PD) and its corresponding supply modulator to the controller at different time intervals, as a result, the shared controller, which is composed of an ADC, a MCU and a DAC, can cope with the information of different micro rings at different time period.

\subsection{SIMO DC-DC Converter}
\begin{figure}[t!]
    \centering
    \includegraphics[width=\linewidth]{pic/TDM/DCDC.pdf}
    \caption{The TDM scheme applying in DC-DC converter}
    \label{fig:DCDC}
\end{figure}
\begin{figure}[t!]
    \centering
    \includegraphics[width=0.8\linewidth]{pic/TDM/DCDC-timing.pdf}
    \caption{Timing diagram of the SIMO DC-DC converter}
    \label{fig:DCDC-timing}
\end{figure}
The DC-DC converters are usually applied in the SoC power management circuit for its high converting efficiency. However an off-chip inductor is required for every DC-DC converter. When multiple power domains exists, which is quite normal in a SoC circuit, multiple DC-DC converters are implemented conventionally. This not only increase the number of required on-chip pads, but also may bring about great noise to the circuit. In \cite{SIMODCDC,SIMODCDC1,SIMODCDC2} an single inductor multiple output (SIMO) DC-DC buck converter is proposed, as is shown in Fig\ref{fig:DCDC}.

Both sub-converter a and converter b are operating in DCM mode, and Fig\ref{fig:DCDC-timing} explicitly interprets the working principle. $\Phi a$ and $\Phi b$ are two complementary phase signals which indicate the regulating period. When $\Phi a = 1$, the sub-converter A is connected to the inductor, and the current through the inductor ramps up at $D_{1a}T$ and ramp down at $D_{2a}T$. Similar current behavior happens again when $\Phi b =1$, and the current is assigned to different sub-converters without affecting each other. 

\subsection{the proposed TDM DLDO}
\begin{figure}[t!]
    \centering
    \includegraphics[width=\linewidth]{pic/TDM/TDMDLDO.pdf}
    \caption{the proposed TDM DLDO}
    \label{fig:TDMDLDO}
\end{figure}
Contrary to the analog counterpart, in the Digital LDO design, the control circuit takes up a great percentage of the whole chip area, even greater than the power stage. The control circuit is usually composed of shift registers, and each power MOS array unit needs a shift register stage. So it would be very promising if we can share the control circuit between different outputs like the TDM MRMs and the SIMO DC-DC converters.

Fig.\ref{fig:TDMDLDO} shows our proposed TDM DLDO, and the working principle is very simple and similar to the TDM MRM. The mux select different outputs to the comparator at certain time periods. Then the comparator compares the output voltages with the reference, and the comparison result serves as the input signal of the controller, which finally determines the states of the power PMOS array. Multiple outputs share the single control circuit at different time period, so the proposed DLDO is operating at TDM mode.

\section{Architecture and working principle of the proposed TDM DLDO}
\subsection{Architecture}
The architecture of the proposed TDM DLDO is shown in Fig.\ref{fig:TDMDLDO}. It mainly composes of a comparator, a output MUX, an analog-assisted loop, two groups of latches, namely lat\_0 and lat\_1 in the diagram, two groups of PMOS arrays, in each group the size of the ``coarse'' PMOS array unit is 9 times larger than that of the ``fine'' PMOS array unit, and two groups of shift register, which are coarse and fine shift registers. The shift registers determine the states of the PMOS arrays and can be asynchronously set/reset by the special set/reset block.  

In the whole diagram, not only the comparator, but also the analog-assisted loop and the coarse and fine shift registers are shared the by two outputs. The latches serve as the buffer stage, which also exist in many other DLDO designs\cite{AALDO,AALDO1}. Apart from that, only an additional multiplexer, the ``outputMux'' is added to the circuit, which scarcely bring any additional area to the whole chip. In other word, we almost double the efficiency of the control circuit in terms of area.
\subsection{Delay switching technique}
\begin{figure}[t!]
    \centering
    \includegraphics[width=\linewidth]{pic/struc/timing.pdf}
    \caption{Timing diagram of the proposed design}
    \label{fig:timing}
\end{figure}
\begin{figure}[t!]
    \centering
    \includegraphics[width=0.9\linewidth]{pic/struc/flowChart.pdf}
    \caption{Flowchart of the delay switching process}
    \label{fig:flowchart}
\end{figure}
\begin{figure*}[t!]
    \centering
    \includegraphics{pic/struc/sche.pdf}
    \caption{Structure of the proposed TDM DLDO}
    \label{fig:diag}
\end{figure*}
\begin{figure}[t!]
    \centering
    \includegraphics[width=0.7\linewidth]{pic/struc/peak.pdf}
    \caption{Peak detector}
    \label{fig:peak}
\end{figure}
In the TDM scheme design, cross coupling is one of the most salient problems. In the SIMO DC-DC converter proposed in \cite{SIMODCDC}, the sub-converters work in DCM mode, so there is a inherit guard period  between different regulating periods when the current running through the inductor stays zero. And in the TDM micro ring proposed in\cite{wangzhicheng}, a guard period is also inserted between different processing periods to avoid the cross coupling between different outputs. In DLDO, the PMOS arrays are controlled by the shift registers. If we share the shift registers without protection, the states of other PMOS array may be loaded during the switching period and great cross-channel coupling occurs. To solve this problem, we propose the delay switching technique.

Fig.\ref{fig:timing} shows the timing diagram of our design, ``clk'' is the system clock and ``path'' is the path switching signal. The ``sele'' signal generates a pulse at every rising and falling edges of the ``path''. And ``P0/1SeleClk'' are the derivative signal from ``path'' and ``sele'', which drives the latch groups. “D0/1” stands for the states of the PMOS array, and “D” stands for the outputs of the shift register. 

The working principle of the delay switching technique is best interpreted with the reference of the flowchart shown in Fig.\ref{fig:flowchart}. At every switching instance, the ``sele'' signal generates a pulse, in other word, ``sele'' = 1 during this period. If ``path'' = 1 at that time, meaning that the control circuit is to be connected to the power PMOS array 1, then the power PMOS array 0 is latched by ''lat\_0'', so the output voltage at path 0 maintains if the load does not change. Meanwhile, the asynchronous set/reset block load the previously latched states of the array 1 to the shift registers. After all the data are loaded and the pulse at ``sele'' signal disappears, then the power PMOS array 1 is officially connected to the shift registers. And the shift registers starts to control the states of the power PMOS array according to the comparison results until the next switching instance happens. As we load its previous states before we connect one path to the control circuit, no cross-channel coupling should exist.

\subsection{The shared analog-assisted loop}
When the load switch from light to heavy, the power PMOS array may not able to provide enough current, and an undershoot occurs at the output. Normally, the shift register should turn on the PMOS array units to provide more current. However, as is shown in Fig.\ref{fig:aaloop}a, the array unit remains high until the clock arrives and the shift register transfer the comparison result to the array. As a result, the undershoot may be very large in the DLDO. To solve this problem, \cite{AALDO} proposed the analog-assisted loop. The working principle is clear, as is shown in Fig.\ref{fig:aaloop}b, if we can feedback the undershoot to the gate of the array unit, the unit can be at the half-open state before the clock arrive, so the undershoot is greatly reduced. To realize it, an RC loop connect the output and the ground of the inverter buffer. So whenever an undershoot occurs, the buffer can sense it and transfer it to the array instantly.

However, the on-chip capacitors and resisters consume lots of area, but AA loop only takes effect when the load changes, in other word, it stay idle at most of the time. So if we can share the AA loop between different paths, the area efficiency can be further increased. Fig.\ref{fig:sharedAAloop} shows the working principle of the shared AA loop. The high pass network senses the undershoot at the confluent output and transfers it to the $V_{SSB}$ node. 
The multiplexers connected to latch groups are controlled by the ``path'' signal. If one path is selected, the multiplexer connects $V_{SSB}$ to the corresponding latch group and another multiplexer connect GND to the other latch group. So the shared AA loop only take effects on the active path, thus greatly improves the usage efficiency.
\begin{figure}[t!]
    \centering
    \includegraphics[width=\linewidth]{pic/struc/aaloop.pdf}
    \caption{Comparison between with AA loop and without AA loop}
    \label{fig:aaloop}
\end{figure}
% \begin{figure}[t!]
%     \centering
%     \includegraphics[width=0.9\linewidth]{pic/struc/latch.pdf}
%     \caption{The latch structure appied in ''lat\_0'' and ''lat\_1''}
%     \label{fig:latch}
% \end{figure}
\begin{figure}[t!]
    \centering
    \includegraphics[width=0.9\linewidth]{pic/struc/sharedAAloop.pdf}
    \caption{Shared AA loop}
    \label{fig:sharedAAloop}
\end{figure}
\subsection{Coarse-fine switching}
In shift register based DLDO, there is always a trade off between accuracy and speed. To obtain a faster transient response, the size of the power PMOS array units need to be large. However, a larger units may bring about larger ripple when the output settles and one unit toggles between ON and OFF. If we choose a smaller PMOS array unit, the output ripple may be small, but there needs to be much more clock periods to settle, so the transient response may be poor. To solve this problem, many existing topologies have adopted the coarse-fine switching techniques\cite{pipeline,coarse-fine,AALDO1}. The coarse-fine switching controller applied in our design is shown in Fig.\ref{fig:peak}, output is compared with two boundaries, and the comparison results are processed by a NAND gate then to control the clock signal. If the output is beyond two boundaries, the coarse-fine switching controller select ``lclk'', and the output of ``sclk'' is zero. So the coarse shift register is driven by a clock signal and fine shift register does not work. As a result, the larger units in the PMOS array turn on and off and the output is regulated at a faster speed. Once the output reach within the boundaries, the coarse-fine switching controller then select ``sclk'' and the coarse shift register stops changing and fine shift register starts to control the smaller units of the PMOS array, as a result ripple is smaller when the output settles.
\subsection{Asynchronous set/reset}
The asynchronous set/reset circuit for the shift register is shown in Fig.\ref{fig:Asyn}. Each unit in the circuit only control a single bit of the shift register, so there will be m units of the asynchronous set/reset circuit in total. This circuit is only active at the switching period between different paths and is controlled by the ``path'' and ``sele'' signal. The ``path'' signal select the inactive path in the DLDO and transfer its signals to inputs of the AND gates. When path switches, an pulse appears at the ``sele'' signal. As a result, the signals at the input of the AND gates transfers to the corresponding bits of the shift register and set/reset the shift registers to its previous states before the inactive path is to be connected to the comparator.
\begin{figure}[t!]
    \centering
    \includegraphics[width=\linewidth]{pic/struc/AsynSet.pdf}
    \caption{Asynchronous set/reset circuits}
    \label{fig:Asyn}
\end{figure}
\subsection{Maximum regulated outputs}
In a stable system, the load of the LDO changes at a regular frequency, we can define the load change period as $\rm T_{load}$. Considering the best case that load change happens right after the ``sele'' signal falls down. As is shown in Fig.\ref{fig:sele}, firstly, undershoot (overshoot) occurs, then the shift register alternate the states of the PMOS array to accommodate the load, after the current provided by PMOS array meets the load requirement, the output returns to the regulated voltage. We can define the response time of the DLDO as $\rm T_{response}$, the maximum number of regulated output paths of the TDM DLDO is determined by $\rm T_{load}$, the duration of the ``sele'' signal $\rm T_{sele}$ and the response time $\rm T_{response}$. The maximum regulated outputs are:
\begin{equation}
\rm N \leqslant \frac{\rm T_{load}}{\rm 2\times(T_{sele}+T_{response})}
\end{equation}
The response time of the DLDO is determined by the size of the PMOS array unit, the extent of load variation and clock frequency. Generally a larger PMOS array, smaller load variation and faster clock frequency may reduce the response time. However, larger PMOS array may lead to larger output ripple and a faster clock frequency may increase the dynamic power consumption. The influence of duration of the ``sele'' signal will discussed later. Normally, the response time $\rm T_{response}$ is on the order of tens of clock period.

Note that when the output reaches the nearby of the reference, either turning on or off one PMOS array unit may change the comparison result and limit cycle oscillation (LCO) occurs. In other word, the DLDO only regulate the output at the switching status, and disconnect the PMOS array from the shift register at the stable status may even be beneficial to the performance.
\begin{figure}[t!]
    \centering
    \includegraphics[width=0.9\linewidth]{pic/consid/sele.pdf}
    \caption{The response process of DLDO when load changes}
    \label{fig:sele}
\end{figure}

\section{Circuit implementation and considerations}
\subsection{``sele'' signal duration}
The ``sele'' signal in  our design has two functions. Firstly, it can modulate the ``path'' signal and generates ``p0SeleClk'' and ``p1SeleClk'' which control the active period of the two paths. In some way, this signal serves as a guard period between the two processing periods, so the duration time of ``sele'' needs to be long enough to avoid cross-talk between two paths. Secondly, ``sele'' is the enabled signal of the asynchronous set/reset circuit which load the states of the next path to the shift register before it is connected to the loop. So the duration of ``sele'' needs to be longer than the response time of shift register. However, from another perspective, the ``sele'' signal duration should not be too long as well. During the active period of the``sele'' signal, both paths are not connected to the control loop of the DLDO. On the one hand, this decreases efficiency of the circuit, on the other hand, there is risk that load change may happens during the ``sele'' period, so large undershoot may appear if the path is not connected to the loop. In our design the pulse duration of ``sele'' signal is about half period of the clock signal. 
\subsection{``path'' signal duration}
``path'' signal determines which path is to be connected to the control loop. We only demonstrate the two paths case in this paper, so only one ``path'' signal is needed, in which HIGH select one path, and the LOW selects the other one. If $n$ ``path'' signals are provided, the maximum numbers of paths that the TDM DLDO can regulate is $2^n$. There are two mainly perspectives to choose the ``path'' signal period. First, we can determine it by the load change frequency. if load change drastically at one path, it needs to be connected to the controlled loop for longer time, otherwise large undershoot or overshoot may occurs. But if the load changes not so frequently at one path, we can just latch the PMOS array states and connect the control circuit to other paths to improve the efficiency of control circuit. However, in this scheme, we need to know how the load changes on each paths in advance. Another way is to  simply increase the ``path'' signal frequency to avoid the large undershoot and overshoot situations. If the ``path'' signal is fast enough, all paths can be regarded as connected to the control loop simultaneously, and the undershoot or overshoot is similar to the one path situation. Nevertheless, assuming the regulating period for all paths is $T$, and the response time of each path is $T_{res}$, then:
\begin{align}
\rm \frac { T}{2^n} &> T_{res}\\
\rm T &> 2^nT_{res}
\end{align} 
\subsection{R and C selection in shared AA loop}
The high-pass network constructed by the resistor and coupling capacitor in the AA loop sense the undershoot and overshoot at output and transfer it to the gate of PMOS array units, so the path can response to load change asynchronously. The effect of the AA loop is determined by the time constant of the high-pass network. Generally, a large time constant helps the output to recover to the regulated voltage faster. As capacitor takes up much larger area in the layout, we usually choose a relatively large resistor to increase the time constant value. However, different from the design in \cite{NANDbasedAAloop,AALDO1}, we need to share the AA loop between different paths. As a result, if the time constant is too large, the voltage at $V_{SSB}$ may not return to GROUND yet when the path switching happens, and it would cause the PMOS array at another path response to this change. In other word, cross-channel coupling happens between different paths. So there is a trade off between cross-channel coupling and response time.
\subsection{Sense amplifier based comparator and D-type flip flop}
Comparator and D-type flip flop are two core circuits in the DLDO design. Comparators are usually sense amplifier based dynamic comparator. The schematic that adopted in our paper is shown in Fig.\ref{fig:comp}\cite{comparator}. It has two stages, the pre-amplify stage consisting of M8 - M12, and the regenerate stage which is constructed by M1 - M7. When ``clk'' is low, M1 and M12 turn off and M8 and M9 turn on, the supply charge the gate of M4 and M7 through M8 and M9. As both M4 and M7 turn on, ``out+'' and ``out-'' are dropped to GROUND and turn on M2 and M3. When ``clk'' becomes high, M1 turns on, so the supply charge the ``out+'' and ``out-'' nodes initially. As M12 turns on the pre-amplify stage, the difference at M10 and M11 is amplified. The outputs of pre-amplify stage controls the charge speed of ``out+'' and ``out-''. Once one of the nodes first reaches the threshold voltage, the regenerate stage forms a positive feedback, and this node become HIGH and the other one becomes LOW.

D-type flip flop is another structure which plays an important role in the DLDO design, but hardly mentioned in the literature. The D-type flip flop adopted in our design is also a sense amplifier based structure as is shown in Fig.\ref{fig:DFF}, and the working principle of it is similar to the dynamic comparator.  When clock signal is low, M10 is closed,so the sense amplifier stop working, and M1 and M4 charge the nS and nR to high, as a result, the output latch keep the previous input data. When clock signal becomes high, M1 and M4 are closed, and M10 is open, the sense amplifier acts like a positive feedback network, input D and nD transfer the data to the output of the sense amplifier, and the latch keep the result until next clock falling edge arrives. The structure transfer data only at the clock edge, so it works like a D-type flip flop.
\begin{figure}[t!]
    \centering
    \includegraphics[width=0.9\linewidth]{pic/struc/comp.pdf}
    \caption{Schematic of sense amplifier based dynamic comparator}
    \label{fig:comp}
\end{figure}
\begin{figure}[t!]
    \centering
    \includegraphics[width=\linewidth]{pic/struc/DFF.pdf}
    \caption{Schematic of sense amplifier based flip flop}
    \label{fig:DFF}
\end{figure}

\section{Measurement results and comparison}
\begin{figure}[t!]
    \centering
    \includegraphics[width=0.9\linewidth]{pic/resu/layout.pdf}
    \caption{The layout of the TDM DLDO}
    \label{fig:layout}
\end{figure}
\begin{figure}[t!]
    \centering
    \includegraphics[width=\linewidth]{pic/resu/output.pdf}
    \caption{Output voltage of the proposed TDM DLDO}
    \label{fig:output}
\end{figure}
The TDM DLDO was fabricated in UMC 130 nm 1P8M CMOS process. The layout of the proposed design was shown in Fig.\ref{fig:layout}. The area of the core circuit is about 0.1855$\rm mm^2$. As is shown in Fig.\ref{fig:layout}, the shift register and control circuit takes up much more area than power PMOS array. As we can share the shift register almost at no cost, the total chip area of TDM DLDO is greatly reduced when comparing with two DLDOs. Fig.\ref{fig:output} shows the output voltages of the proposed TDM DLDO. The red line stands for confluent output of the two paths, and output 0 and output 1 are the output voltages at two paths respectively. It is clearly shown that the output 0 and output 1 coincides with the output line at different time period and no cross-channel coupling exists at the switching instance. The largest undershoot at both paths is about 200 mV. Fig.\ref{fig:curr} shows the load current changes at two paths. Load change happens when their corresponding path is connected with the control circuit. The edge time of the load change is about 120 ns, and load change from 2 mA to 80 mA at path 0 and from 2 mA to 90 mA at path 1. Fig.\ref{fig:gnd} shows the $V_{SSB}$ and analog ground voltage of the latch groups. The red line stands for the $V_{SSB}$ voltage of the shared AA loop, and the green and blue line stands for the analog GROUND voltage of the latch groups at two paths. Similar to the confluent output voltage, the analog GOUND voltage at two paths coincide with $V_{SSB}$ at different time period, and the cross-channel coupling between paths is negligible. In table I, we summary the main performance of our proposed TDM DLDO.
\begin{figure}[t!]
    \centering
    \includegraphics[width=\linewidth]{pic/resu/loadCurr.pdf}
    \caption{Load current of the two paths in TDM DLDO}
    \label{fig:curr}
\end{figure}
\begin{figure}[t!]
    \centering
    \includegraphics[width=\linewidth]{pic/resu/gnd.pdf}
    \caption{$\rm V_{SSB}$ and analog ground of latch groups}
    \label{fig:gnd}
\end{figure}
\section{Conclusion}
In this paper, we presents a novel DLDO which can share the control circuit and the shift registers between two paths. By applying the delay switching and shared AA loop techniques, we greatly reduces the total chip area and avoid the cross-channel coupling between different paths. We verify our idea through the experiment results, and it shows that the performances at two paths of the proposed TDM DLDO is comparable to the state-of-art design but chip area is greatly reduced. As the control circuit becomes more and more complicated and takes up more and more chip area, this technique provide a very promising technique to save the total chip area of the power management network. In addition, this technique can be extended to regulate more paths and reduce the total chip area even further.
\bibliographystyle{ieeetr}
\bibliography{05ref}
\end{document}


